\section{Símbolos Especiales}
\label{sec:simbolosespeciales}

% introduccion a los simbolos especiales
En Director, varios de los simbolos especiales permiten hacer uso de las
confección gráficas de UML para la confección del modelo, por ejemplo, se
tienen varios símbolos que tienen una contraparte en las palabras reservadas
para la descripcion de algun método/atributo. En el \texttt{Cuadro
\ref{c:s_especiales}} se presentan los símbolos que componen al lenguaje y una
breve explicación de cada uno de ellos.

\begin{table}[ht]
\centering
\caption{Símbolos especiales}
\begin{tabular}{c p{9cm} }
\hline
	Símbolo & Descripción \\ \hline
	+ & Visibilidad de un método/atributo, equivalente a public\\
	- & Visibilidad de un método/atributo, equivalente a private\\
  \# & Visibilidad de un método/atributo, equivalente a protected\\
	/ & Visibilidad de un método/atributo, equivalente a derivate\\
$\sim$ & Visibilidad de un método/atributo, equivalente a package\\
	: & Permite la asignación de un tipo de dato a un método/atributo\\
$\#\#$ & Comentarios de línea dentro del modelo \\
$\#\{$ & Inicio comentarios multilínea dentro del modelo\\
$\}\#$ & Fin comentarios multilínea dentro del modelo\\
$\# +$ & Definición de metainformación para el modelo\\
$//$ & Comentario de línea para el código resultante del modelo\\
$/*$ & Inicio comentarios multilínea para el codigo resultante del modelo\\
$*/$ & Fin comentarios multilínea para el codigo resultante del modelo\\ \hline
\end{tabular}
\label{c:s_especiales}
\end{table}
