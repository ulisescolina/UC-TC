\section{Palabras Reservadas}
\label{sec:palabrasreservadas}

Aquí se describirán las palabras utilizadas como tokens en el lenguaje
propuesto, estas permiten, desde el establecimiento de relaciones
entre entidades de un modelo, hasta la habilidad de indicar el tipo de devolución
de un método. A continuación se proporciona una lista de palabras reservadas
con su correspondiente explicación:

\begin{description}[align=right,labelwidth=2.5cm]

\item [class] Esta palabra se utiliza como token para la definición de una clase,
	es un componente de alto nivel dentro del lenguaje.

	Ejemplo:
		\begin{lstlisting}
		class Persona {
			...
		}
		\end{lstlisting}

\item [abstract] Token que permite designar a un componente como abstracto.

	Ejemplo:
		\begin{lstlisting}
		abstract class Persona {
			...
		}
		\end{lstlisting}

\item [relationships] Palabra utilizada para establecer relaciones inter-clase
	dentro de un modelo, esta es una entidad de alto nivel en el lenguaje que
	permite la relación mediate \textit{herencia}, \textit{asociación}, etc,
	entre clases de un modelo.

	Ejemplo:
		\begin{lstlisting}
		...
		relationships {
			...
		}
		\end{lstlisting}


\item [void] Token que permite establecer el tipo de la devolución de un
	método, éste indica que el método no hace ninguna	devolución.

	Ejemplo:
		\begin{lstlisting}
		  ...
			<visibilidad> metodo():void
			...
		\end{lstlisting}

\item [integer] Token que permite indicar el tipo de dato devuelto por un metodo, o
	el tipo de dato de un atributo. Hace referencia al tipo de
	datos entero; equivalente en un lenguaje como Java, \texttt{int}.

	Ejemplo:
		\begin{lstlisting}
		  ...
			<visibilidad> atributo:integer
			<visibilidad> metodo():integer
			...
		\end{lstlisting}

\item [double] Token que permite indicar el tipo de dato devuelto por un metodo, o
	el tipo de dato de un atributo. Hace referencia al tipo de
	datos doble; equivalente en un lenguaje como Java, \texttt{double}.

	Ejemplo:
		\begin{lstlisting}
		  ...
			<visibilidad> atributo:double
			<visibilidad> metodo():double
			...
		\end{lstlisting}

\item [float] Token que permite indicar el tipo de dato devuelto por un metodo, o
	el tipo de dato de un atributo. Hace referencia al tipo de
	datos con presición flotante; equivalente en un lenguaje como Java,
	\texttt{float}.

	Ejemplo:
		\begin{lstlisting}
		  ...
			<visibilidad> atributo:float
			<visibilidad> metodo():float
			...
		\end{lstlisting}

\item [long] Token que permite indicar el tipo de dato devuelto por un metodo, o
	el tipo de dato de un atributo. Hace referencia al tipo de
	datos largo; equivalente en un lenguaje como Java, \texttt{long}.

	Ejemplo:
		\begin{lstlisting}
		  ...
			<visibilidad> atributo:long
			<visibilidad> metodo():long
			...
		\end{lstlisting}

\item [boolean] Token que permite indicar el tipo de dato devuelto por un metodo, o
	el tipo de dato de un atributo. Hace referencia al tipo de
	datos booleanos; equivalente en un lenguaje como Java, \texttt{boolean}.

	Ejemplo:
		\begin{lstlisting}
		  ...
			<visibilidad> atributo:boolean
			<visibilidad> metodo():boolean
			...
		\end{lstlisting}

\item [string] Token que permite indicar el tipo de dato devuelto por un metodo, o
	el tipo de dato de un atributo. Hace referencia al tipo de
	datos de cadena de caracteres ; equivalente en un lenguaje como Java, \texttt{String};
	equivalente en C, \texttt{char[N]}.

	Ejemplo:
		\begin{lstlisting}
		  ...
			<visibilidad> atributo:string
			<visibilidad> metodo():string
			...
		\end{lstlisting}

\item [char] Token que permite indicar el tipo de dato devuelto por un metodo, o
	el tipo de dato de un atributo. Hace referencia al tipo de
	datos de caracter; equivalente en un lenguaje como Java, \texttt{char}.

	Ejemplo:
		\begin{lstlisting}
		  ...
			<visibilidad> atributo:char
			<visibilidad> metodo():char
			...
		\end{lstlisting}

\item [list] Token que permite indicar el tipo de dato devuelto por un metodo, o
	el tipo de dato de un atributo. Hace referencia a una
	estructura de colección reminiscente a las listas; equivalente en un lenguaje
	como Java, \texttt{List}.

	Ejemplo:
		\begin{lstlisting}
		  ...
			<visibilidad> atributo:list<tipo>
			<visibilidad> metodo():list<tipo>
			...
		\end{lstlisting}

\item[set] Token que permite indicar el tipo de dato devuelto por un metodo, o
	el tipo de dato de un atributo. Hace referencia a una
	estructura de colección reminiscente a los conjuntos; equivalente en un lenguaje
	como Java, \texttt{Set}.

	Ejemplo:
		\begin{lstlisting}
		  ...
			<visibilidad> atributo:set<tipo>
			<visibilidad> metodo():set<tipo>
			...
		\end{lstlisting}

\item [public] Esta palabra permite establecer la visibilidad de un elemento
	dentro del modelo, este elemento puede ser un atributo o una clase.

	Ejemplo:
		\begin{lstlisting}
		  ...
			public atributo:<tipo>
			public metodo():<tipo>
			...
		\end{lstlisting}

\item [private] Esta palabra permite establecer la visibilidad de un elemento
	dentro del modelo, este elemento puede ser un atributo o una clase.

	Ejemplo:
		\begin{lstlisting}
		  ...
			private atributo:<tipo>
			private metodo():<tipo>
			...
		\end{lstlisting}

\item [protected] Esta palabra permite establecer la visibilidad de un elemento
	dentro del modelo, este elemento puede ser un atributo o una clase.

	Ejemplo:
		\begin{lstlisting}
		  ...
			protected atributo:<tipo>
			protected metodo():<tipo>
			...
		\end{lstlisting}

\item[derivate] Esta palabra permite establecer la visibilidad de un elemento
	dentro del modelo, este elemento puede ser un atributo o una clase.

	Ejemplo:
		\begin{lstlisting}
		  ...
			derivate atributo:<tipo>
			derivate metodo():<tipo>
			...
		\end{lstlisting}

\item [package] Esta palabra permite establecer la visibilidad de un elemento
	dentro del modelo, este elemento puede ser un atributo o una clase.

	Ejemplo:
		\begin{lstlisting}
		  ...
			package atributo:<tipo>
			package metodo():<tipo>
			...
		\end{lstlisting}

\item [@id] Esta palabra se utiliza como un modificador en el ambito de
	los atributos de clase, este permite brindar información extra acerca del
	atributo en cuestión, en específico, este indica que el atributo es parte del
	identificador de la clase que lo posee.

	Ejemplo:
		\begin{lstlisting}
		  ...
			<visibilidad> atributo:<tipo>{@id}
			...
		\end{lstlisting}

\item [@readOnly] Esta palabra se utiliza como un modificador en el ambito de
	los atributos de clase, este permite brindar información extra acerca del
	atributo en cuestión, en específico, este indica que el atributo solo podra
	ser accedido para lectura una vez se haya guardado.

	Ejemplo:
		\begin{lstlisting}
		  ...
			<visibilidad> atributo:<tipo>{@readOnly}
			...
		\end{lstlisting}

\item [@sequence] Esta palabra se utiliza como un modificador en el ambito de
	los atributos de clase, este permite brindar información extra acerca del
	atributo en cuestión, en específico, este indica que el valor que tome el
	atributo sigue una secuencia.

	Ejemplo:
		\begin{lstlisting}
		  ...
			<visibilidad> atributo:<tipo>{@sequence}
			...
		\end{lstlisting}

\item [@unique] Esta palabra se utiliza como un modificador en el ambito de
	los atributos de clase, este permite brindar información extra acerca del
	atributo en cuestión, en específico, este indica que el atributo no tiene
	duplicados.

	Ejemplo:
		\begin{lstlisting}
		  ...
			<visibilidad> atributo:<tipo>{@unique}
			...
		\end{lstlisting}

\end{description}
