\section{Método}
\label{sec:comentario}
Al igual que los atributos, los metodos, son componentes dentro del diagrama de
clases que ayudan a la manipulación de los datos, estos brindan el
comportamiento que es propio al objeto. El lenguaje presentado permite definir
métodos de la siguiente manera:

\begin{lstlisting}[basicstyle=\footnotesize\ttfamily]
  <metodo>::=<visibilidad><nombre>"("<parametro>")" ":"<tipo>
  <metodo>::=<visibilidad><nombre>"("<parametros>")" ":"<tipo>
\end{lstlisting}

En donde se puede decir que \texttt{parametro(s)} se define de la siguiente
manera:

\begin{lstlisting}[basicstyle=\footnotesize\ttfamily]
  <parametro> ::= <nombre> ":" <tipo>
\end{lstlisting}

\begin{lstlisting}[basicstyle=\footnotesize\ttfamily]
  <parametros> ::= <parametro> | <parametros>
\end{lstlisting}

