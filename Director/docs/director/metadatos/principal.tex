\section{Metadatos}
\label{sec:metadatos}

La naturaleza que rodea a la propuesta, presenta custiones que no se pueden
solucionar en el modelo dado a que no son inherentes al mismo, para dar una
solución a esta situación se decidió la inclusión de otro tipo de elementos al
lenguaje, además de los que representan a los elementos utilizados en el
diagrama de clases de UML, estos elementos son denominados \texttt{Metadatos}.

Los metadatos permiten dar directivas al lenguaje de que se debe realizar con
el modelo, ejemplo:

\begin{itemize}
\item Lenguaje de salida, este brinda al lenguaje cual debe ser el
\texttt{lenguaje de salida} del modelo.
\item Estructura de salida, este brinda ordenes al lenguaje de si las clases
resultantes deben estar separadas en diferentes archivos (obligatorio para
lenguajes como Java) o en un solo archivo (Lenguajes como Python permiten
esto).
\end{itemize}

El hecho de que deba existir otro tipo de cuestiones en el archivo que describe
al modelo, que no necesariamente tiene que ver con el modelo, hace que también
sea necesario agregar algún mecanismo que permita comunicar al lenguaje donde
inicia y donde termina el modelo. Esto se realiza con los siguientes literales,
y utilizando uno de los simbolos ya mencionados en el \texttt{Cuadro
\ref{c:s_especiales}, Sección {\ref{sec:simbolosespeciales}}}.

\begin{lstlisting}[caption={Director - Inicio y Fin del Modelo},
label=lst:drtiniciofin]
	#+BEGINDRT
	  ... Descripcion
    ...    del
		...   Modelo
	#+ENDDRT
\end{lstlisting}

Como se aprecia en el \texttt{Fragmento {\ref{lst:drtiniciofin}}}, la descripción
del modelo con todos los elementos descritos anteriormente están envueltos
en los literales \texttt{BEGINDRT} y \texttt{ENDDRT} antecedidos por uno de los
símbolos especiales \texttt{\#+}, de la misma forma, los tokens para indicar
cuestiones como el lenguaje que se debe generar o la estructura de salida,
son precedidos por el mismo simbolo especial, y se conforman de diferentes
secciones que permiten la configuración de diferentes aspectos de lo que se
obtiene en la salida, se brindará un ejemplo, pero cabe destacar que la
descripcion de todas las secciones posibles para la configuración no se detalla
en el presente documento.

\begin{lstlisting}[caption={Director - Metadatos ejemplos}, numbers=left,
label=lst:metadatoej]
	#+DRT.CONF.O.LANG=java
	#+JAVA.CONF.O.AUTHOR=Ulises
	#+JAVA.CONF.F.STRUCT=split
\end{lstlisting}

A continuación se brinda una explicación de los ejemplos expuestos en el
\texttt{Fragmento \ref{lst:metadatoej}} primeramenta separando los diferentes
componentes	en áreas de interes, luego comentando lo que cada área pretende
represenar:

\begin{itemize}
	\item \texttt{Linea 1}: en primer lugar se tiene \texttt{DRT}, éste indica el
		lenguaje para el cual se estará seteando la configuración, el ejemplo
		concreto indicaría alguna directiva al lenguaje Director de como tratar al
		modelo, lo siguiente \texttt{CONF} indica que se trata de una
		configuración, luego se tiene \texttt{O} el cual se desprende de la palabra
		\textit{Output} haciendo referencia a la salida del programa y finalmente,
		se tiene \texttt{LANG}, lo cual pasa a ser una propiedad que tomara el
		valor de `java' de esta forma se le esta diciendo ``\textit{Para lo que
		tenga que ver con la manipulacióñ de modelos mediante el lenguaje Director,
		establecer la configuracón de lenguaje para la salida del mismo como
		java}''.
	\item \texttt{Linea 2}: aquí se tiene \texttt{JAVA} al principio, por lo
		tanto lo que sigue será para darle directivas a las salidas que estén en
		lenguaje Java, luego se tiene \texttt{CONF} lo cual indica que se estará
		tratando de una sección de configuración, despues se tiene \texttt{O}
		nuevamente, este viene de output, y finalmente se tiene un atributo para la
		sección que se denomina \texttt{AUTHOR}, éste permitirá a la herramiena
		establecer el autor que estará trabajando en el código generados.
	\item \texttt{Línea 3}: éste nuevamente inicia con \texttt{JAVA} lo que
		indica que se darán indicaciones que tienen que ver con el lenguaje Java,
		dentro de este, es de interés la categoría configuración (\texttt{CONF}) y
		aquí, en la configuración interesa la parte ya ``física'' (por la
		\texttt{F}, derivado de ``\textit{File}''), del archivo, ya
		se trata de un archivo con codigo generado en su interior, aqui el atributo
		que se esta estableciendo se denomina \texttt{STRUCT} haciendo referencia a
		la estructura con la que se generará el proyecto, en este caso ``split''
		significaría que las clases estarán en archivos separados, una clase por
		cada archivo.
\end{itemize}


