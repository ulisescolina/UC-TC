\pagenumbering{arabic}
\chapter{Estado de la Cuestión}%
\label{sec:estadodelacuestion}

\section{Introducción}
\label{sec:introduccion}
% Posibles ventajas que aporta el uso de UML
En el campo del desarrollo de software el advenimiento de notaciones como UML
generaron un nuevo paradigma para la implementación de software, si bien esta
notación no es una idea nueva, por el hecho de que tiene sus inicios en la decada de los
1990 siendo la idea principal del mismo, el modelado de sistemas de software en
varios niveles de abstracción  \cite{chaudron2017}, además considerándose por
estudios como la herramienta de notación para modelado de software dominante
\cite{aldaeej2016} aunque usado predominantemente de manera informal y que las
actividades realcionadas al ámbito tienden a ser
más fáciles en un desarrollo con un enfoque en el modelado \cite{forward2010}
con la capacidad de además de tener beneficios dentro del entorno de desarrollo
dado al nivel de expresión que se tiene en las
notaciones gráficas comparadas con las formas textuales de expresión
\cite{rumpe2004}.

% Cuestiones por las cuales el uso de UML no siempre sucede
% dentro del desarrollo de Software
A pesar del resultado de los estudios mencionados, algunas encuestas realizadas
en los mismos muestran que los desarrolladores prefieren centrar el trabajo en vastas
cantidad de líneas de código, raramente mirando diagramas, y mucho menos
confeccionándolos o editándolos, posibles razones para este
comportamiento, en principio antagónico, en donde los efectos positivos de UML pueden
verse reducidos \cite{bente2006} se describen a continuación:
\textit{la no-factibilidad de realizar ingeniería inversa en código heredado},
\textit{costo de entrenamiento del equipo} y \textit{los requerimientos no
estaban distribuidos en unidades funcionales del sistema}.

En el mismo estudio, se consultó cuáles son las cuestiones de trabajar con el
desarrollo orientado a modelos que lo hacen difícil de tratar o sean la razón
por la cual no se modele mucho. Las razones principales fueron:
\begin{enumerate}
	\item \textit{Los modelos se desactualizan con respecto al código},
		es decir no se encuentra un flujo de trabajo que contemple a ambas
		actividades tal que estas dos avancen a la par, sino que se deben
		realizar por separado, causando así que el modelo quede desactualizado
		con frecuencia.
	\item \textit{Los modelos no son intercambiables fácilmente}, aquí se habla de la
		dificultad de, nuevamente, encontrar un flujo de trabajo que permita el paso de
		los modelos de forma más ``natural''.
	\item \textit{Las herramientas para modelado son pesadas}, haciendo alusión a los
		recursos computacionales en los cuales se debe incurrir para utilizar dichas
		herramientas.
		\item \textit{El código generado desde una herramienta no es del agrado del
			desarrollador}.
		\item \textit{Crear y modificar un modelo es lento}.
	\end{enumerate}

Además de la encuesta anterior, también se realizó
una centrándose en el desarrollo \textit{centrado en la codificación}, los
siguientes fueron los mayores inconvenientes identificados por los desarrolladores.

\begin{enumerate}
	\item Ver de ``un pantallazo'' todo el diseño es un problema, una forma de
		transmitir la dificultad que se tiene de brindar una visión a un alto nivel
		del proyecto, lo cual lleva al tercer punto de la lista.
	\item Comportamiento del sistema difícil de entender.
	\item La calidad del código se degrada con el tiempo.
	\item La reestructuración del proyecto es muy compleja.
	\item Los cambios de código pueden producir bugs.
\end{enumerate}

% Luego describir sus bases teóricas
%  DSL, DSML, MDx y la importancia de cada una
\section{Desarrollo de Software Dirigido por Modelos}
\label{sec:mdsd}
El desarrollo de software dirigido por modelos (MDSD, de ahora en más, por sus
siglas en inglés) como se menciona en \cite{mdsd} , surge a partir de la
popularización de UML;
el uso del mismo, sin embargo, solamente se restringía a la confección de
documentación, debido a motivos ya mencionados. El acercamiento de que ofrece
MSDS es enteramente diferente, la parte ``dirigido'' (Driven, en MDSD),
enfatiza la importancia y el rol central que le da este paradigma al modelo
éste ya no solamente constituye
la documentación del software, sino que tambien es considerado igual a código;
incluso es aplicable en campos de alta especialidad debido a una de las
características del acercamiento: realización de abstracciones específicas de
dominio y hacer estas abstracciones accesibles a través del modelado. Ésta
característica permite la automatización de la implementación de código,
haciendo posible a la vez el incremento en la productividad y la mantenibilidad
de los sistemas.

Para que se pueda aplicar el concepto de ``modelo específico de dominio'' se
deben tener en cuenta tres requerimientos:
\begin{itemize}
	\item \textit{Lenguajes de dominio específico}, para la formulación de modelos.
	\item Lenguajes que puedan expresar transformaciones ``modelo-código''.
	\item \textit{Compiladores, generadores o transformadores}, para generar el codigo
		ejecutable en varias plataformas.
\end{itemize}

\section{Arquitectura Dirigida por Modelos}
\label{sec:arquitectura_dirigida_por_modelos}
La arquitectura dirigida por modelos (MDA, por sus siglas en ingles) es una
iniciativa introducida por el \textit{Object Management Group}, con el propósito de
brindar una forma estandarizada para la especificación e interoperabilidad de
sistemas basado en el uso formal de modelos \cite{poole2001}, en el núcleo de
MDA se encuentran otros estándares implementados por el OMG: \textit{the Unified
Modeling Language} (UML), \textit{Meta Object Facility} (MOF), \textit{XML
Metadata Interchange}
(XMI) y el \textit{Common Warehouse Metamodel} (CWM). Al igual que el MDSD, MDA, ubica
los modelos de sistemas en el núcleo del problema de interoperabilidad lo cual
hace que la implementación del sistema sea independiente de la tecnología.

La OMG, promueve MDA como un \textit{marco de trabajo arquitectónico para el
desarrollo de software}, el cual está construido alrededor de un número de
especificaciones detalladas de la misma organización que son usadas ampliamente
en la comunidad de desarrolladores.

Esto puede hacer pensar que $MDA=MDSD$, lo cual sería correcto hasta cierto
punto, en principio, el acercamiento de MDA es similar al de MDSD, pero difiere
en detalles, por ejemplo, éste, tiende a ser mas restrictivo, enfocándose
primariamente en lenguajes de modelado basados en UML \cite{mdsd}.


\section{Lenguaje Específico de Dominio}
\label{sec:dsl}
Un Lenguaje Específico de Dominio (DSL) es un lenguaje de software el cual se
especializa en abarcar un problema dentro de la ingeniería en particular, las
cuales son características para un dominio de aplicación, éste, se basa en
abstracciones alineadas a este dominio y provee una sintáxis propicia para
aplicar estas abstracciones de forma efectiva \cite{sobernig2015}.

La implementación de un DSL permite mitigar ciertos aspectos que se vieron como
desventaja al inicio, algunos mencionados en \cite{nguyen2014} incluyen la
reutilización de código, promueve la legibilidad y el entendimiento debido al alto
nivel de abstracción del mismo, permite a que usuarios con nivel bajo en
programación la creación de modelos para programas siempre y cuando estos
posean el conocimiento del dominio, más verificaciones en la sintáxis y
semántica que un lenguaje de modelado general; aunque también se enfatizan
desventajas que estos insertan en el desarrollo, curva de aprendizaje
necesaria y la falta de personas letradas en el DSL, ya que es más probable que
las personas sepan como resolver los problemas adoptando un lenguaje de
propósito general el cual ya conocen.

Algunos autores definen las características deseables de un DSL
\cite{mazanec2012}:
\begin{itemize}
	\item La capacidad de describir todo el software.
	\item La capacidad de describir varios niveles de abstracción.
	\item Legibilidad y Simplicidad del lenguaje.
	\item Expresiones no Ambiguas.
	\item Soporte e Integrabilidad.
\end{itemize}

% Aca se puede seguir con el paper de Umplification, demostrando que se
% implemento una herramienta que ayuda a las problematicas planteadas.
\section{Ejemplos de Implementaciones}

\subsection{Umple}
Un caso con bastante aceptación dentro del desarrollo de software llevado
mediante modelos es la herramienta Umple {\cite{umple-official}}, la cual tiene
como objetivo mitigar varios de los inconvenientes enumerados anteriormente, con
respecto a la renuencia de los equipos de trabajo para el modelado y al
problema que cada desarrollador ve en un desarrollo centrado en la
codificación \cites{aldaeej2016}{garzon2014}, esto a través de la aplicación de
refactorizaciones al código lo cual da como resultado un programa equivalente
al original, con el agregado de que este puede ser renderizado y editado
mediante herramientas UML \cite{lethbridge2010}, siguiendo diferentes
paradigmas dentro del desarrollo de software tales como el uso de un Lenguaje
DSL (\texttt{Sección \ref{sec:dsl}}).

\subsection{Telosys}
El análisis realizado en \cite{telosys} lo muestra como una herramienta simple
(que utiliza cuestiones tales como los DSL para la creación del modelo),
provee la habilidad de la generación de código teniendo como base un modelo,
el cual se le provee a la misma mediante una interfaz de línea de comandos, su
objetivo es proveer una alternativa al clásico ``Primero el UML'' dentro del
desarrollo, esto significa, que en vez de invertir el tiempo del
desarrollador al inicio del proyecto documentando diagramas UML, teniendo como
principio lo que se expuso en la \texttt{Sección
\ref{sec:arquitectura_dirigida_por_modelos}} se
tiene que el modelo \textbf{es} el código, es decir, que los límites que separan
la documentación de la codificación se desvanecen.

\section{Objetivos}
\label{sec:problema}
Aquí se presenta una herramienta que aplica los
conceptos vistos en la \texttt{Sección \ref{sec:mdsd}} referentes a MDSD,
focalizando el uso de modelos UML, tales restricciones determinan que se siguen
los patrones relacionados a MDA como se acotó en la
\texttt{Sección \ref{sec:arquitectura_dirigida_por_modelos}}, como se menciona
en la sección para MSDS, para la implementación de tales técnicas, uno de los
requisitos es la definición de un Lenguaje Específico de Dominio, (tratado en
la \texttt{Sección \ref{sec:dsl}}).

Teniendo esto en claro se procede a la declaración de los objetivos y estos
son:

Lograr una herramienta que sea útil en el desarrollo de software dirigido por
modelos UML.
Director deberá permitir modelar un dominio a través del código --utilizando un
lenguaje específico de dominio--; el resultado será la generación de un código en
un lenguaje de programación orientado a objetos con la estructura equivalente a
dicho dominio, con sus atributos, métodos y relaciones correspondientes.

Además, frente a una situación de refactorización, o cambios en el dominio,
Director tendrá que permitir la actualización del modelo, con su correspondiente
regeneración de código, sin sobreescribir la lógica que ya haya sido incluida.
Los lenguajes de programación soportados dependerá de los plugins que se puedan
agregar según la necesidad, y lo desarrollado por la comunidad.

El desarrollo de Director, así como de los plugins determinados, será bajo
licencia de tipo CopyLeft.

\section{Alcance}%
\label{sec:alcance}
Se define la sintáxis, léxico y estructura del lenguaje Director; así como algunas
de sus características; esto es independiente del lenguaje en el cual se obtendrá el
código generado finalmente.

En esta instancia Director será capaz de expresar únicamente el modelo y relaciones
correspondientes a un diagrama de clases.

Los ejemplos de generación de código serán con Java/Pyton; sin embargo, como ya se
explicó anteriormente, Director no estará limitado únicamente a este último.

\section{Líneas Futuras de Investigación}%
\label{sec:lineas_futuras_de_investigacion}
Si bien todo lo expuesto referente al desarrollo guiado por modelos -como ser metodologías,
herramientas y otras cuestiones- data de los años 90, su popularidad y auge están creciendo
cada vez más, sobre todo entre los grupos que buscan nuevas técnicas en el desarrollo de sistemas
que presentan nuevos desafíos. Por lo tanto, es nuestro propósito lograr una herramienta no-final,
sino una que siga evolucionando constantemente según las necesidades que vayan surgiendo.

A pesar de que Director considera un amplio espectro de cuestiones sobre un Lenguaje Específico
de Dominio, y del modelado UML, somos conscientes de que quedan aún muchas cosas afuera. Dichas
cuestiones quedan pendientes de seguir trabajando, y son:
\begin{itemize}
  \item Como se mencionó en el Alcance del trabajo, Director es capaz de expresar únicamente un
    diagrama de clases; por lo tanto, queda analizar la posibilidad de incluir otros diagramas de
    UML en el lenguaje -Máquinas de estado, Diagramas de Secuencia, Diagramas de Paquetes, Diagramas
    de Actividades, etc- que brinden la posibilidad de modelar mejor el contexto del sistema, con sus
    funcionalidades, estados y relaciones; esto permitirá ampliar las capacidades de generación de código,
    y mejorar la efectividad y eficiencia en el producto final, automatizando no sólo el modelo, sino
    también la lógica de negocio.
  \item Generación de los diagramas UML correspondientes; es decir, exportar del lenguaje a diagramas
    como imágenes.
  \item Generación automática de documentación.
\end{itemize}
