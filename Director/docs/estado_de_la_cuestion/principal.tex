\pagenumbering{arabic}
\chapter{Estado de la Cuestión}%
\label{sec:estadodelacuestion}

\section{Problemática}
\label{sec:problematica}
% Posibles ventajas que aporta el uso de UML
En el campo del desarrollo de software el advenimiento de notaciones como UML
generaron un nuevo paradigma para la implementación de software, si bien esta
notación no es una idea nueva, por el hecho de que tiene sus inicios en la decada de los
1990 siendo la idea principal del mismo, el modelado de sistemas de software en
varios niveles de abstracción  \cite{chaudron2017}, además considerandose por
estudios como la herramienta de notación para modelado de software dominante
\cite{aldaeej2016} aunque usado predominantemente de manera informal y que las
actividades realcionadas al ámbito tienden a ser
más fáciles en un desarrollo con un enfoque en el modelado \cite{forward2010}
con la capacidad de además de tener beneficios dentro del entorno de desarrollo
dado al nivel de expresión que se tiene en las
notaciones gráficas comparadas con las formas textuales de expresión
\cite{rumpe2004}.

% Cuestiones por las cuales el uso de UML no siempre sucede
% dentro del desarrollo de Software
A pesar de lo mencionado, encuestas realizadas en uno de los estudios que trata
a UML como una notación dominante dentro del desarrollo de software realizada años
atras, muestran que desarrolladores prefieren centrar el trabajo en vastas
cantidad de líneas de código, raramente mirando diagramas, y mucho menos
confeccionándolos o editándolos, posibles razones para esto se pueden para este
comportamiento se descubre en \cite{bente2006} en donde se identifican cuatro
razones por las cuales los efectos positivos de UML pueden verse reducidos,
aunque para el presente trabajo se puede hacer hincapié en tres de los
mencionados:
\textit{la no-factibilidad de realizar ingenieria inversa en codigo heredado},
\textit{costo de entrenamiento del equipo} y \textit{los requerimientos no
estaban distribuidos en unidades funcionales del sistema}.

En el mismo estudio, se consultó cuáles son las cuestiones de trabajar con el
desarrollo orientado a modelos que lo hacen difícil de tratar o sean la razon
por la cual no se modele mucho. Las razones principales fueron:
\begin{enumerate}
\item \textit{Los modelos se desactualizan con respecto al código},
es decir no se encuentra un flujo de trabajo que contemple a ambas
actividades tal que estas dos avancen a la par, sino que se deben
realizar por separado, causando así que el modelo quede desactualizado
con frecuencia.
\item \textit{Los modelos no son intercambiables fácilmente}, aquí se habla de la
dificultad de, nuevamente, encontrar un flujo de trabajo que permita el paso de
los modelos de forma más ``natural''.
\item \textit{Las herramientas para modelado son pesadas}, haciendo alusión a los
recursos computacionales en los cuales se debe incurrir para utilizar dichas
herramientas.
\item \textit{El código generado desde una herramienta no es de mi agrado}.
\item \textit{Crear y modificar un modelo es lento}.
\end{enumerate}

Además de la encuesta anterior, también se realizó
una centrándose en el desarrollo \textit{centrado en la codificación}, los
siguientes fueron los mayores inconvenientes identificados por los desarrolladores.

\begin{enumerate}
	\item Ver de ``un pantallazo'' todo el diseño es un problema, una forma de
		transmitir la dificultad que se tiene de brindar una visión a un alto nivel
		del proyecto, lo cual lleva al tercer punto de la lista.
	\item Comportamiento del sistema difícil de entender.
	\item La calidad del código se degrada con el tiempo.
	\item Muy dificil reestructurar el proyecto.
	\item Cabios de codigo pueden implementar bugs.
\end{enumerate}



\section{Lenguaje Específico de Dominio}
\label{sub:introestadocuestion}
Un lenguaje especifico de dominio (DSL) es un lenguaje de software el cual se
especializa en abarcar un problema dentro de la ingeniería en particular, las
cuales son características para un dominio de aplicación, éste, se basa en
abstracciones alineadas a este dominio y provee una sintaxis propicia para
aplicar estas abstracciones de forma efectiva \cite{sobernig2015}.


