\documentclass{article}
\renewcommand\refname{Referencias}
\usepackage{graphicx}
\graphicspath{{../imagenes/}}
\begin{document}
\title{\includegraphics{unam-logo.png}\\Trabajo Pr\'actico N\'umero 3\\Teor\'ia de la Computaci\'on}
\author{Ulises C. Ramirez}
\date{30 Abril 2018}
\maketitle
\pagenumbering{gobble}
\newpage

% === Inicio del cuerpo del documento. ===%
\section*{Acerca del documento}
\label{sec:s1}
En el paquete subido al Aula Virtual, podra encontrar una serie de imagenes en formato SVG (las cuales podr\'an ser abiertas en cualquier navegador web, \textit{probado con el navegador Chromium}) que compilar\'an lo que se requiere por la c\'atedra para el {\bf Trabajo Pr\'actico 3}, en el apartado {\bf Poruesta de Trabajo}.

Cabe mencionar que no se agrega ning\'un resumen de los temas mencionados en la consigna por el hecho de que estas cuestiones ya fueron abordadas en el \textbf{Trabajo Pr\'actico Numero 1}, y consider\'e que ser\'ia redundante volver a realizar lo mismo en este documento. Sin embargo, agrego un mapa conceptual que resume el tema de \textbf{Dise\~{n}o} tratado en el mismo trabajo, por lo que el presente trabajo puede complementar al TP1.


\section*{Problema con C++ un estudio de caso en el dise\~{n}o de lenguajes}
\label{prob1}
En la realizacion del mapa conceptual de este apartado, me v\'i obligado a utilizar \cite{comppth} dado a que este titulo solamente lo tengo en ese libro, me encontr\'e con que, de la \textbf{p\'agina 20} del libro que es donde inicia el apartado \textbf{3.5}, se saltea 2 hojas, por lo que unicamente pude acceder a una parte de la lectura.
\newline
\newline
\newline

\textbf{\textsc{Importante}}: el trabajo pr\'actico se llevo a cabo teniendo en cuenta unicamente el material descrito en \cite{teolengprog} , dado a que la copia que se tiene de \cite{comppth} era ilegible en algunos tramos, ademas a esta \'ultima le faltaban paginas, o las mismas estaban muy desordenadas, como se mencion\'o anteriormente.

\pagenumbering{arabic}


% ==========Bibliografia===================================
\newpage
\begin{thebibliography}{9}
	\bibitem{comppth} 
	\textsc{Kenneth C. Louden}
	\textit{Lenguajes de Programaci\'on: Principios y Pr\'acticas}, Segunda Edici\'on.
	San Jose State University.

	\bibitem{teolengprog} 
	\textsc{Fernando L\'opez Ostenero, Ana Mar\'ia Garc\'ia Serrano}
	\textit{Teor\'ia de los Lenguajes de Programaci\'on}. 
	Editorial Universitaria Ram\'on Areces.
\end{thebibliography}
\end{document}
